% ______________________________________________________________________________
%
%   1DV600 - Software Technology
%   Assignment 3 -- "Testing"
%
%  Author:  Jonas Sjöberg
%           Linnaeus University
%           js224eh@student.lnu.se
%           https://github.com/jonasjberg
%
%    Date:  2017-03-03 -- 2017-03-05
%
% License:  Creative Commons Attribution 4.0 International (CC BY 4.0)
%           <http://creativecommons.org/licenses/by/4.0/legalcode>
%           See LICENSE.md for additional licensing information.
% ______________________________________________________________________________


% ______________________________________________________________________________
\section{Task 1 -- Test Plan}

\paragraph{Instructions}\label{task-1-instructions}
from the course Wiki\cite{1dv600:lab3:instructions}:

\begin{quote}
  To get an overall structure to test, the first task is to create a test plan.
  Use the discussion in the lectures for a structure and additionally get some
  inspiration from http://epf.eclipse.org/wikis/openup/ ­­ you will not find a
  template as such there, but lots of information on what you need to include.
  You can also use other sources for inspiration and search for templates, as
  long as they seem to be valid. Make use of your requirements from the
  previous assignment and possibly additional material from Assignment 2. Don’t
  forget that the quality requirements will require a special treatment.
  Identify objectives, objects and create a work breakdown structure using test
  management processes, dynamic test processes, and possibly static processes
  for objects and objectives. Specify suitable techniques for testing in the
  processes you have identified. Write down your reflections on creating a test
  plan with about 100 words.
\end{quote}



% ______________________________________________________________________________
\subsection{Test Plan}\label{task-1a}
% TODO: Everything!

% * Find and use Test Plan template.
% * Make use of the requirements (+additional material) from previous
%   assignment.
% * IMPORTANT: Quality requirements will require a special treatment.
% * Identify objectives, objects
%
% * Create a work breakdown structure using
%     * Test Management Processes
%         * ("possibly" Static Test Processes "for objects and objectives")
%         * Dynamic Test Processes
%     * Test Management Integration
%         * ("possibly" Static Test Processes "for objects and objectives")
%         * Dynamic Test Processes
%
% * Specify sutable techniques for testing in the identified processes.

\subsubsection{Setting up the Test Tools}
There is no doubt that the de-facto standard for testing Java codebases is
JUnit\cite{tools:junit}, a testing library and framework that I have used
extensively in the past.
Most popular Java IDEs have some kind of support for integrating various
third-party testing frameworks/tools. 
This is often integrated test-runners and viewers for test results and other
test output, which makes continous testing a lot easier to setup.

It is important to be able to run the tests or suite of tests quickly --
if test execution is slow and development must halt for some time; which could
be a couple of seconds (smaller projects) or an entire night (Microsoft Windows).
% TODO: Add citation for above

Fast test suite execution encourages developers to run the test suite often,
ideally directly after every single functional modification of the code.

If the test suite is too slow, developers might put off running the tests.




The project testing will be using JUnit for all test code. 


\subsection{High-level Test System Planning}
I usually configure my IDE, \texttt{Intellij IDEA Ultimate 2016.3.4}, to
automatically run the tests or test suite when the source files are modified on
disk. The test execution is almost instant for most trivial projects on my main
desktop workstation.

This project will use the same methodology, continous execution of fast tests.


% TODO: Dynamic Test Processes

% TODO: Test Management Integration
% TODO: Dynamic Test Processes
%
% TODO: sutable techniques for testing in the identified processes.



\subsection{Reflections}\label{task-1c-reflect}
% TODO: Reflection on creating a test plan in 100 words.

