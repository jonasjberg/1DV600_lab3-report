% ______________________________________________________________________________
%
%   1DV600 - Software Technology
%   Assignment 3 -- "Testing"
%
%  Author:  Jonas Sjöberg
%           Linnaeus University
%           js224eh@student.lnu.se
%           https://github.com/jonasjberg
%
%    Date:  2017-03-03 -- 2017-03-05
%
% License:  Creative Commons Attribution 4.0 International (CC BY 4.0)
%           <http://creativecommons.org/licenses/by/4.0/legalcode>
%           See LICENSE.md for additional licensing information.
% ______________________________________________________________________________


% ______________________________________________________________________________
\section{Task 1 -- Test Plan}

\paragraph{Instructions}\label{task-1-instructions}
from the course Wiki\cite{1dv600:lab3:instructions}:

\begin{quote}
  To get an overall structure to test, the first task is to create a test plan.
  Use the discussion in the lectures for a structure and additionally get some
  inspiration from \url{http://epf.eclipse.org/wikis/openup/} ­­ you will not
  find a template as such there, but lots of information on what you need to
  include.  You can also use other sources for inspiration and search for
  templates, as long as they seem to be valid. Make use of your requirements
  from the previous assignment and possibly additional material from Assignment
  2. Don't forget that the quality requirements will require a special
  treatment.  Identify objectives, objects and create a work breakdown
  structure using test management processes, dynamic test processes, and
  possibly static processes for objects and objectives. Specify suitable
  techniques for testing in the processes you have identified. Write down your
  reflections on creating a test plan with about 100 words.
\end{quote}



% ______________________________________________________________________________
\subsection{Test Plan}\label{task-1a}
% TODO: Everything!

% * Find and use Test Plan template.
% * Make use of the requirements (+additional material) from previous
%   assignment.
% * IMPORTANT: Quality requirements will require a special treatment.
% * Identify objectives, objects
%
% * Create a work breakdown structure using
%     * Test Management Processes
%         * ("possibly" Static Test Processes "for objects and objectives")
%         * Dynamic Test Processes
%     * Test Management Integration
%         * ("possibly" Static Test Processes "for objects and objectives")
%         * Dynamic Test Processes
%
% * Specify suitable techniques for testing in the identified processes.

\subsubsection{Setting up the Test Tools}
There is no doubt that the de-facto standard for testing Java codebass is
JUnit\cite{tools:junit}, a testing library and framework that I have used
extensively in the past.
Most popular Java IDEs have some kind of support for integrating various
third-party testing frameworks/tools. 
This is often integrated test-runners and viewers for test results and other
test output, which makes continuous testing a lot easier to setup.

The project testing will be using JUnit for all test code. 

It is important to be able to run the tests or suite of tests quickly -- if
test execution is slow and development must halt for some time; which could be
a couple of seconds (smaller projects) or during an entire night (MS Windows).
% TODO: Add citation for above

Fast test suite execution encourages developers to run the test suite often,
ideally directly after every single functional modification of the code.

If the test suite is too slow, developers might put off running the tests.

I usually configure my IDE, \texttt{Intellij IDEA Ultimate 2016.3.4}, to
automatically run the tests or test suite when the source files are modified on
disk. The test execution is almost instant for most trivial projects on my main
desktop workstation.

This project will use the same methodology, continuous execution of fast tests.

% TODO: Finish Dynamic Test Processes!

\subsection{High-level Test System Planning}
\subsubsection{Dynamic Test Processes}
The codebase will be tested at a low-level. This requires that the code is
written such that components can be used and tested in isolation.
It is most helpful to divide the system into smaller parts that can be tested
in isolation without being intertwined and dependent on other parts.

My private projects are mostly written in a style that works well with
extensive TDD-testing of especially the most critical systems. These are often
parsers and pattern matching routines, where unit testing really shines at
identifying incorrect behaviour and manual verification is time consuming and
error-prone. The code base is written such that systems allow easy access
and testing in isolation. The exact details depend on context, a pattern-matching
routine can likely be written in a functional style; the test code need not
handle any kind of setup or initialization routine to ensure that the test
is done in a well-defined state. 

Some systems are inherently better suited for a object-oriented programming
style.  This requires some extra work, the objects/classes/modules to be tested
should be designed to allow for testing in isolation. Some components depend on
some other system interface, this is often handled by substituting ``mock''
objects in place of the dependencies. 

An example could be testing a component that requires a database connection.
This could be done by presenting the object with an interface that matches that
of an actual database, but use a much simpler mock database during tests. This
has numerous advantages and eliminates variables that would otherwise introduce
noise in the tests. This could be the `HTTP` connection in the database
example.

We just want to test the functionality of our code.

% This project could be divided into a user-facing layer; the Java program
% that we are developing.  connects to 

\subsubsection{Test Management Integration}
The project testing will primarily use the JUnit integration features provided by the IDE.
Additional software is needed for more sophisticated integration -- bug
tracking, team collaboration tools, time management, etc.

A popular choice is \texttt{Jira}\cite{tools:jira} by Atlassian. However, I would much
prefer an open source tool like GitHub Issues\cite{tools:github-issues} or
Launchpad \cite{tools:launchpad}.

\subsubsection{Dynamic Test Processes}
% TODO: Dynamic Test Processes

\subsubsection{Dynamic Test Processes}
% TODO: Suitable techniques for testing in the identified processes.

\subsection{Reflections}\label{task-1c-reflect}
% TODO: Reflection on creating a test plan in 100 words.


