% ______________________________________________________________________________
%
%   1DV600 - Software Technology
%   Assignment 3 -- "Testing"
%
%  Author:  Jonas Sjöberg
%           Linnaeus University
%           js224eh@student.lnu.se
%           https://github.com/jonasjberg
%
%    Date:  2017-03-03 -- 2017-03-05
%
% License:  Creative Commons Attribution 4.0 International (CC BY 4.0)
%           <http://creativecommons.org/licenses/by/4.0/legalcode>
%           See LICENSE.md for additional licensing information.
% ______________________________________________________________________________


% ______________________________________________________________________________
\section{Task 2 -- Test Cases}

\paragraph{Instructions}\label{task-1-instructions}
from the course Wiki\cite{1dv600:lab3:instructions}:

\begin{quote}
    Take the two use cases that you identified in assignment 2 and create test
    cases for them. Describe what it is you are going to test, how you are
    going to do it and what outcome you expect from different tests. In
    addition, write down your reflections with about 100 words.
\end{quote}



% ______________________________________________________________________________
\subsection{Test Case for Use Case 1 -- Adding a Book}\label{task-2a}
This use case was described in the previous assignment.

\subsubsection{Use case 1 -- Scenario}\label{task-2a}
The first use case describes adding a new book to the library.


\subsubsection{Use case 1 -- Tests}
For this test, the initial state of the book library must be known. We want
to test adding a book, so we must know which books are already present.
The books are identified with unique identifiers, book equality has been
previously defined.
The Book that is added during the test is created somehow, and this creation
itself should also be tested. However, it is not obvious that testing this in
the same code base or source file.
After the Book has been added, we compare the initial known state to the
expected final state; initial state with the additional added book.

The tests will follow a common JUnit pattern that sets up initial conditions
prior to each test, runs the test and then tears down the test environment.
This is done for every test case so that the initial conditions can be
controlled by the tester.

An initial \texttt{setup}-routine will create the objects and define some
known starting state. We will need a Book-object to add and a Library-object
to add the book to.

We then add enough test cases to confirm expected behaviour, which means test
expected common behaviour and probably some unexpected ``edge-cases'' or error
conditions. 

An example test case could be:

Adding book A to an empty library should result in a library containing the book A.

Testing adding another book B in the same way would not improve test coverage
as the programs would behave just the same.

Another test case should be adding books to a library that already contains a
book, adding a book to a library that already contains that book, and so on.


% ______________________________________________________________________________
\subsection{Test Case for Use Case 2 -- List Books}\label{task-2b}
This use case was described in the previous assignment.

\subsubsection{Use case 2 -- Scenario}
The user Gibson visits the book library application website to view all books
he has previously entered. The complete list of books appear at once as he
visits the website. The listing is the default starting page and so Gibson does
not have to interact with the application to view the listing.


\subsubsection{Use case 2 -- Tests}
This test could be implemented both at a lower level, testing the responses
returned by the underlying database and the commands sent through the 
web framework that eventually talks to the browser, that finally renders the
actual graphics to be seen.

The test could also be done at a higher level, as a kind of integration test,
where we only care about testing the output that matters to the user, which is
what actual graphics eventually gets rendered by the browser.

For this project, I will only test communications between the different 
components and classes in the Java code I've written myself.


\subsection{Reflections}\label{task-2-reflect}
I have done a lot of TDD and unittesting in various previous projects.  Some
tests can be very difficult to set up so that the test results actually reflect
back on whether the code under test \emph{actually} performs as intended or
not. It is difficult to have full test coverage\cite{wiki:code-coverage}, and
even if you manage to acheive a very high coverage, there might be parts of the
code that is simply impossible to test. This is very context-depdendent and
vary with programming languages, executing platform, complications caused by
multi-threading, parallellizing, etc. There could be some state of the program
that would require testing the underlying interpreter or virtual machine. Even
if the entire project is built on open source code, some complex states are
often still practically unreachable due to budget constraints.

I usually apply testing to where it makes sense and the time required to
implement the testing is justifiable. I have tried all-out TDD, striving for
100\% coverage, and I have concluded that a lot of code makes little sense to
test, while other parts really demands extensive testing if the project is to
be taken at all seriously.


