% ______________________________________________________________________________
%
%   1DV600 - Software Technology
%   Assignment 3 -- "Testing"
%
%  Author:  Jonas Sjöberg
%           Linnaeus University
%           js224eh@student.lnu.se
%           https://github.com/jonasjberg
%
%    Date:  2017-03-03 -- 2017-03-05
%
% License:  Creative Commons Attribution 4.0 International (CC BY 4.0)
%           <http://creativecommons.org/licenses/by/4.0/legalcode>
%           See LICENSE.md for additional licensing information.
% ______________________________________________________________________________


% ______________________________________________________________________________
\section{Task 4 -- API Tests}

\paragraph{Instructions}\label{task-4-instructions}
from the course Wiki\cite{1dv600:lab3:instructions}:

\begin{quote}
  Perform additional test on the API for the system by creating tests for all
  the resources that are available in the API specification. You will find some
  examples of how this is done in the file PingResourceTest . All tests will be
  executed when you perform a vagrant reload and if a test fails, the building
  will stop. Lastly, write down your reflections on testing APIs with about 100
  words.
\end{quote}



% ______________________________________________________________________________
\subsection{API Tests}\label{task-4}
The API tests are included in the source code for this report.


\subsection{Reflections}\label{task-4-reflect}
Testing APIs is very important as this code is the ``glue'' and interface
between our own codebase and third party services. If a third party changes
their API, all users must be sure to update the code that depends on said API.

Also, if the third party accidentally introduces unwanted behaviour or bugs
into their code, we could detect it by checking the actual API responses
against what we expect.

It is probably worthwhile to design the application such that there is very
clear separation between the code that communicates with API and the rest of
the code, so that one could substitute a dummy API in place of the actual API
without modifying anything. This would be an extra layer of abstraction,
wrapping the API calls.

This would also influence how the testing code can be written, making offline
testing and mocking easier do.

